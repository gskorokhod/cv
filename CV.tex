%-------------------------
% Resume in Latex
% Author : Sourabh Bajaj
% Website: https://github.com/sb2nov/resume
% License : MIT
%------------------------

\documentclass[letterpaper,11pt]{article}

\usepackage{latexsym}
\usepackage[empty]{fullpage}
\usepackage{titlesec}
\usepackage{marvosym}
\usepackage[usenames,dvipsnames]{color}
\usepackage{verbatim}
\usepackage{enumitem}
\usepackage[pdftex]{hyperref}
\usepackage{fancyhdr}

\pagestyle{fancy}
\fancyhf{} % clear all header and footer fields
\fancyfoot{}
\renewcommand{\headrulewidth}{0pt}
\renewcommand{\footrulewidth}{0pt}

% Adjust margins
\addtolength{\oddsidemargin}{-0.375in}
\addtolength{\evensidemargin}{-0.375in}
\addtolength{\textwidth}{1in}
\addtolength{\topmargin}{-.5in}
\addtolength{\textheight}{1.0in}

\urlstyle{same}

\raggedbottom
\raggedright
\setlength{\tabcolsep}{0in}

% Sections formatting
\titleformat{\section}{
  \vspace{-4pt}\scshape\raggedright\large
}{}{0em}{}[\color{black}\titlerule \vspace{-5pt}]

%-------------------------
% Custom commands
\newcommand{\resumeItem}[2]{
\item\small{
    \textbf{#1}{ #2 \vspace{-2pt}}
  }
}

\newcommand{\resumeSubheading}[5]{
  \vspace{-1pt}
\item
  \begin{tabular*}{0.97\textwidth}{l@{\extracolsep{\fill}}r}
    \textbf{#1} & #2 \\
    \textit{\small#3} & \textit{\small #4} \\
  \end{tabular*}
  \par{\small{#5}\vspace{-2pt}}
}

\newcommand{\resumeSubItem}[2]{\resumeItem{#1}{#2}\vspace{-4pt}}

\renewcommand{\labelitemii}{$\circ$}

\newcommand{\resumeSubHeadingListStart}{
\begin{itemize}[leftmargin=*]}
    \newcommand{\resumeSubHeadingListEnd}{
  \end{itemize}}
\newcommand{\resumeItemListStart}{
\begin{itemize}}
    \newcommand{\resumeItemListEnd}{
  \end{itemize}\vspace{-5pt}}

%-------------------------------------------
%%%%%%  CV STARTS HERE  %%%%%%%%%%%%%%%%%%%%%%%%%%%%

\begin{document}

%----------HEADING-----------------
\begin{tabular*}{\textwidth}{l@{\extracolsep{\fill}}l}
  \textbf{{\Large Georgii Skorokhod}} & Email :
  \href{mailto:mail@gskorokhod.com}{mail@gskorokhod.com}\\
  {Fourth year Computer Science student} & Mobile : +44 (0) 7586 882 668 \\
  {} & Githib: \href{https://github.com/gskorokhod}{@gskorokhod}
\end{tabular*}

%-----------EDUCATION-----------------
\section{Education}
\resumeSubHeadingListStart
\resumeSubheading
{University of St. Andrews}{St. Andrews, Scotland, UK}
{Computer Science (BSc); Grade average: 17.5 / 20 (Year 2)}{Sep. 2022 - Present}
{}
\resumeSubheading
{Bauman Engineering School №1580}{Moscow, Russia}
{High School (Physics \& Mathematics track); Grade average: 5.0 /
5.0}{Sep. 2017 - Jun. 2022}
{}
\resumeSubHeadingListEnd

%-----------EXPERIENCE-----------------
\section{Experience}
\resumeSubHeadingListStart
\resumeSubheading
{Intern (Central Technology group)}{Cambridge, UK}
{ARM}{Jun. 2025 - Present}
{ I am currently working as an intern for the Central Technology
  group at ARM, where I am involved in the development of an internal
tool for benchmarking chip performance using a machine learning model. }
\resumeSubheading
{Research Intern (STARIS)}{St. Andrews, Scotland, UK}
{University of St. Andrews}{Jun. 2024 - May 2025}
{ As part of the St Andrews Research Internship Scheme (STARIS), I am
  working on the practical application of constraints programming for
  scheduling and workforce management problems, under the supervision
of Dr. Ozgur Akgun. }
\resumeItemListStart
\resumeItem{Reactive UI with Svelte \& Typescript \\}
{I have developed a reactive web application that gives users an
  intuitive way to enter employee data and availability, manage tasks,
schedule shifts, and define the constraints that need to be satisfied. }
\resumeItem{Geocoding with the OSM API \\}
{To allow users to easily schedule shifts for different physical
  locations, I have developed an integration with the OpenStreetMaps
  API to look up coordinates and addresses, and visualise them on an
interactive map. }
\resumeItem{Constraints modelling \\}
{I have worked on client-side logic to generate an input file to the
  Conjure constraints modelling tool based on the data provided by the
user, including handling custom user-defined constraints. }
\resumeItem{CI and TDD \\}
{I have developed unit tests with Vitest, integration tests with
  Playwright, and automated linting, testing, and deployment using
GitHub Actions.}
\resumeItemListEnd
\resumeSubheading
{Vertically Integrated Project - Constraints Modelling}{St. Andrews,
Scotland, UK}
{University of St. Andrews; Grade achieved: 20 / 20}{Sep. 2023 - May 2024}
{As part of a Vertically Integrated Project (VIP), I have worked with
  a team of students and university staff on the
  \href{https://github.com/conjure-cp/conjure-oxide}{'conjure-oxide'}
  project - a full Rust rewrite of the
  \href{https://github.com/conjure-cp/conjure}{'conjure' constraints
  modelling tool}, focusing on a clean architecture, compile-time
optimisation, and support for incremental solving of constraints problems. }
\resumeItemListStart
\resumeItem{Rust \\}
{Working on this project, I have learned the Rust programming
  language and gained a better understanding of memory management,
concurrency, macros, and the cargo build system.}
\resumeItem{Developing an Essence language compiler \\}
{Essence is a domain-specific language for defining constraints
  problems, used as input for Conjure. I have developed a Rust
  representation of an Essence AST and a
  \href{https://github.com/conjure-cp/conjure-oxide/wiki/Expression-rewriting,-Rules-and-RuleSets}{rewrite
  engine} that simplifies it down to a low-level, solver-specific
  representation.  As part of this, I have developed macros to
auto-generate the necessary code for traversing the AST in a generic way.}
\resumeItem{AGILE \\}
{I have worked alongside a team of students from multiple years,
  using git for version control and collaboration, and following the
AGILE software development methodology. }
\resumeItemListEnd
\resumeSubHeadingListEnd

\section{Extracurricular Activities}
\resumeSubHeadingListStart
\resumeSubheading
{Committee Member for Research}{St. Andrews, Scotland, UK}
{Campaign for Affordable Student Housing (CASH)}{May 2024 - Dec. 2024}
{  }
\resumeItemListStart
\resumeItem{The St. Andrews Rent Map \\}
{One of my main tasks is administering an annual survey of students
  and community members about their housing situation, analysing the
  data, and creating a
  \href{https://www.google.com/maps/d/viewer?mid=1zf78ODx0XEIRSom3jh77OIWpTgcgyjE&usp=sharing}{visual
map} of rental prices around St Andrews.  }
\resumeItem{Research \& Support \\}
{As a CASH committee member, my core mission was conducting research
  to ensure that our campaign
  can operate safely and effectively, as well as raising
  awareness about housing issues and informing students about their
rights as tenants. }
\resumeItem{The St. Candrews Food Bank \\}
{I have taken part in organising donation drives for a
  student-operated food bank to help students and other members of our
community who are struggling with the cost of living. }
\resumeItemListEnd
\resumeSubHeadingListEnd

\section{Projects}
\resumeSubHeadingListStart
\resumeSubItem{\href{https://github.com/gskorokhod/PygameChess}{Pygame
Chess} \\}
{A simple chess game written in Python, complete with an automated
opponent using the MinMax algorithm.}
\resumeSubItem{Linux Homelab \\}
{I am using a Linux server to self-host a file sharing service and
  gaming servers for myself and my friends, automatic regular backups
  of important files from my laptop, and my website at
  \href{https://gskorokhod.com}{gskorokhod.com} (currently under
  development).\\ This has helped me learn Docker, the UNIX command
line, and the basics of web technology. }
\resumeSubItem{Greenbox \\}
{In high school, I have worked with a team of students to build a
  fully automated greenhouse for herbs and small plants, based on the
  Arduino microcontroller. It included climate control, lighting, and a
  configuration UI. Our project has won a bronze medal for the Russian
  delegation at the International Exhibition for Young Inventors
(IEYI-2019) in Jakarta.}
\resumeSubHeadingListEnd

%-------SKILLS------------
\section{Skills}
\resumeSubHeadingListStart
\resumeSubItem{Programming languages}{: Rust, C, Python, Java,
Typescript, JS + HTML + CSS, SQL }
\resumeSubItem{Skills \& Frameworks}{: Git, Docker, UNIX Shell,
Tensorflow, Node.js, Svelte, React, SQL Databases }
\resumeSubItem{Fundamentals}{: Algorithms \& Data Structures,
Networking, OS Programming, Web Development }
\resumeSubItem{Natural languages}{: English (IELTS 8.0), Russian
(native), German (beginner) }
\resumeSubHeadingListEnd

\section{Modules}
\resumeSubHeadingListStart
\resumeSubheading
{CS3102 (Data Communications \& Networks)}{Grade: 16.4 / 20 (Final:
TBA)}{University of St. Andrews}{2024-2025}
{Studied the design and implementation of network protocols such as
  TCP in depth; Learned about the OSI model and the TCP/IP stack;
Developed a reliable data transfer protocol in C using sockets.}
\resumeSubheading
{CS3105 (Artificial Intelligence)}{Grade: 15.5 / 20}{University of
St. Andrews}{2024-2025}
{Studied the fundamentals of AI, including search algorithms,
  knowledge representation, machine learning techniques, and Bayesian
  statistics.
  Implemented and Benchmarked various search algorithms on the Knight's
Tour problem. }
\resumeSubheading
{CS3104 (Operating Systems)}{Grade: 17.9 / 20}{University of St.
Andrews}{2024-2025}
{Studied the various components of operating systems, including
  scheduling, memory management, file systems, and device drivers.
  Implemented a scheduler, a memory allocator, and a copy of the
\verb|ls| user-space utility for a simple kernel written in C++. }
\resumeSubheading
{CS3052 (Computational Complexity)}{Grade: 18.3 / 20 (Final:
TBA)}{University of St. Andrews}{2024-2025}
{Studied the fundamentals of complexity theory, including Turing
  machines, language classes, and reductions. Learned to assess the
space and memory complexity of practical algorithms.}
\resumeSubheading
{CS3050 (Logic \& Reasoning)}{Grade: 15.5 / 20}{University of St.
Andrews}{2024-2025}
{Studied the fundamentals of propositional logic, the predicate
  calculus, and first-order theories.
  Applied practical reasoning techniques and systems, such as
  tableau, to solve practical problems.
  Developed a simple interpreter for the Prolog logic programming
language in Java. }
\resumeSubheading
{$2^{\text{nd}}$ Year Modules}{Grade: 17.5 / 20 (average)}{University
of St. Andrews}{2023-2024}
{Studied the fundamentals of computer networking, web development,
  concurrency, and systems programming. Learned about algorithms,
  data structures, and the fundamentals of computation. Developed a
  multi-user CLI messaging application using TCP sockets in Java, a
  REST API server in Node.js, a reactive drag-and-drop UI for
managing playlists in jQuery, and a C program for evaluating logical formulae.}
% {CS2002 - Computer Systems}{Grade: 17.6 / 20}
% {University of St. Andrews}{2023-2024}
% { I have learned about system calls, inline assembly, memory
%   management and concurrency in C. Furthermore, I have developed a C
%   program for generating the truth tables of logical expressions, and
% learned about the DPLL algorithm. }
% \resumeSubheading
% {CS2001 - Foundations of Computation}{Grade: 17.3 / 20}
% {University of St. Andrews}{2023-2024}
% { I have learned some fundamental concepts in Computer Science,
%   including Finite State Automata, grammars, computational complexity,
%   and Big-O Notation. Moreover, I have practiced implementing trees,
% hash sets and other data structures in Java. }
% \resumeSubheading
% {CS2003 - The Internet and the Web}{Grade: 16.8 / 20}
% {University of St. Andrews}{2023-2024}
% { I have learned about the OSI model and the fundamentals of
%   networking. Applying this knowledge in practice, I have developed a
%   simple client-server multi-user CLI messaging application using TCP
%   sockets in Java, built a REST API server in Node.js, and developed a
% reactive drag-and-drop UI for managing playlists. }
\resumeSubheading
{IE1250 - Mathematics B}{Grade: 17.7 / 20}
{University of St. Andrews}{2022-2023}
{ Covered the
  basics of algebra and calculus, such as limits, integration,
differentiation, series, complex numbers, and linear algebra.  }
\resumeSubHeadingListEnd

%-------------------------------------------
\end{document}
